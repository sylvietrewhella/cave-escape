\documentclass[12pt]{article}
\usepackage{hyperref}

\begin{document}

\begin{titlepage}

\newcommand{\HRule}{\rule{\linewidth}{0.5mm}} % Defines a new command for the horizontal lines, change thickness here

\center % Center everything on the page

%----------------------------------------------------------------------------------------
%	HEADING SECTIONS
%----------------------------------------------------------------------------------------

\textsc{\LARGE Swinburne University}\\[1.5cm] % Name of your university/college

%----------------------------------------------------------------------------------------
%	TITLE SECTION
%----------------------------------------------------------------------------------------

\HRule \\[0.4cm]
{ \huge \bfseries Cave Escape, the Presentation Guide}\\[0.4cm] % Title of your document
\HRule \\[1.5cm]

%----------------------------------------------------------------------------------------
%	AUTHOR SECTION
%----------------------------------------------------------------------------------------

\begin{minipage}{0.4\textwidth}
\begin{flushleft} \large
\emph{Authors:}\\
Jake \textsc{Renzella} % Your name
\end{flushleft}
\end{minipage}
~
\begin{minipage}{0.4\textwidth}
\begin{flushright} \large
\emph{ } \\
Reuben \textsc{Wilson} % Supervisor's Name
\end{flushright}
\end{minipage}\\[4cm]

% If you don't want a supervisor, uncomment the two lines below and remove the section above
%\Large \emph{Author:}\\
%John \textsc{Smith}\\[3cm] % Your name

%----------------------------------------------------------------------------------------
%	DATE SECTION
%----------------------------------------------------------------------------------------

{\large \today}\\[3cm] % Date, change the \today to a set date if you want to be precise

%----------------------------------------------------------------------------------------
%	LOGO SECTION
%----------------------------------------------------------------------------------------

%\includegraphics{Logo}\\[1cm] % Include a department/university logo - this will require the graphicx package

%----------------------------------------------------------------------------------------

\vfill % Fill the rest of the page with whitespace

\end{titlepage}

\section{Introduction to the Cave Escape Presentation.}
Cave Escape is a clone of the popular mobile game Flappy Bird which has been made in
Pascal using SwinGame. This presentation uses snapshots of the games progress as it was developed.

The aim of the presentation is to show the audience the basics of how game development works, just how simple making
a game can be, and how much fun they can have.

Throughout the course of the presentation, the Presenter will present different iterations of the code of the game, in different levels of completion.

All versions of code shown to the auidience will be complete, meaning they are executable. The purpose is to allow the presnter to visualise
with the audience, what different blocks of code actually do to he game.

\subsection{Installation.}
The development envioronment of Swingame using Pascal requires a few tools to be isntalled that the presenter or audience
may not have. Thankfully, there are detailed instructions in the form of videos which run through installing the necessary tools to compile SwinGame.

The presenter will not need these tools to present, this package is precompiled and ready to go, however if the audience would like to alter the source code
for their own versions of the game, they should be directed to the videos on
\href{https://www.youtube.com/playlist?list=PLdVESrjTNUXtU8zclRh9ovhstzWQAY05U}{\underline{YouTube}} and chose the isntallation videos.

\section{}

\end{document}
